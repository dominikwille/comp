\input{../../abv/lib/my}

\newcommand{\myTitle}{\LaTeX - Grundlagen}
\newcommand{\myAuthor}{Dominik Wille}
\newcommand{\myDate}{22 Oktober 2013}
\newcommand{\myTitleImage}{images/typesetting} %This Image is GNU GPL licenced.
\newcommand{\myTitleLeft}{%
   Freie Universität Berlin\\
   Zentraleinrichtung für Datenverarbeitung\\
   Betriebsysteme und Programmieren%
}
\newcommand{\myTitleRight}{%
  Dozent: \\
  Dr.\,Herbert Voß
}

\begin{document}
\myTitlepage
\section{...}
\subsection{...}
\subsection{...}
\subsection{Rundungsfehler}
Rechenoperationen mit reelen zahlen im Computer $\rightarrow$ Rundungsfehler.
\subsubsection{Gleitkommaaritmetik}
Im Vergleich zum Festpunktformat: geringerer Speicherplatzbedarf. \\ \\
$n$-stellige Gleitkommazahl, Basis $B$:
\begin{equation}
  x = \pm \left(0,z,z_2,...,z_n\right)_B \cdot B^E = \pm \sum_{i=1}^n Z_i \cdot Z_i != 0
\end{equation}
(Normalisierte Gleitkommadarstellung)
Exponent: $E e \mathbb{Z} : m <= E <= M$\\
Bsp: $+1234,567 = +(0,1234567)_{10}\cdot 10^4$ \\
($B = 10, n = 7$)
Die Werte $n,B,m,M$ maschienenabhängig (Hardware und Compiler)
Übliche Basen:
\begin{itemize}
\item $B = 2$ (Dualzahlen, im Computer)
\item $B = 8$ (Oktalzahlen)
\item $B = 10$ (Dezimal)
\item $B = 16$ (Hexdezimal)
\end{itemize}
Bsp: binäre Darstellung: 
\begin{align}
(5,0625)_{10} &= 0,50625 \cdot 10^1 \\
&= 1 \cdot 2^2 + 1 \cdot 2^0 + 0 \cdot 2^{-1} + 0 \cdot 2^{-2} + 0 \cdot 2^{-3} + 1 \cdot 2^{-4} \\
  = (101,0001)_2 &= (0,1010001)_2 \cdot 2^{3}
\end{align}
manche Zahlen lassen sich nur schwer als Dualzahlen darstellen:\\
\begin{itemize} 
\item $(3)_{10} = (11)_2$ geht
\item $(0,3)_{10} = 0 \cdot 2^{-1} + 1 \cdot 2^{-2} + ... = (0,010011001...)_2$ geht nicht
\end{itemize}
\textbf{Genauigkeit der Darstellung} \\ \\
23 Stellen $\Roghtarrow$ $11111111111111111111111 = 2^{23} - 1 = 8.388.608$ \\
$\Rightarrow$ 6 Ziffren können unterschieden werden. \\ \\
52 Stellen $2^{52} = 4.503.599.627.370.496$\\
$\Rightarrow$ 15 Stellen können unterschienden werden.
Die größte darstellbare Zahl entspricht der größten Maschienenzahl.
\begin{align}
x_{max} = (0,[B-1][B-1]...[B-1])_B \cdot B^M = (1-B^{-n}) \cdot B^M 
\end{align}
kleinste darstellbare Zahl
\begin{align}
x_{min} = (0,1000000)_B \cdot B^m = (1-B^{-n}) \cdot B^{m-1} 
\end{align}
\begin{center}
  $\Rightarrow$ \textit{Die menge der Maschienenzahlen ist endlich}
\end{center}
\underline{Bsp:}\\
$x_{max} + x_{max} = \infty$ \\
$x_{min} \cdot B^{-1} = 0
\subsubsection{Rundungsfehler}
Beim runden einer Zahl $x$ wird eine Näherung $rd(x)$ unter den Maschienenzahlen geliefert,  so dass der absolute Fehler $\left|x-rd(x)\right|$ minimal ist, der unvermeidbare Fahler ist der Rundungsfehler. Eine $n$-stellige Dezimalzahl im Gleitkommaformat
\begin{align}
  x = \pm (0,z_1,...,z_n)_{10}=rd(x)
\end{align}
hat einen maximalen absoluten Fehler : 
\begin{align}
  \left| x - rd(x)\right| &<= 0,000..005 \cdot 10^E \\
  &= 0,5 \cdot E^{E-n}
\end{align}
, für allgemeine Basis $B$:
\begin{align}
  \left|x-rd(x)\right| <= \frac{B \cdot 1}{2 \cdot B} B^{E-n} = \frac{1}{2} B^{E-n}
\end{align}
\underline{Rundungsfehler werden durch die rechnung getragen!} \\ \\
$n$-stellige Gleitkommaaritmetik: \\
jede einzelne Rechenoperation ($+,-,\times,\div$)wird auf $n+1$ Stellen genau berechnet und dann auf n stellen gerundet. Jedes Zwischenergebnis, nicht Endergebnis! \\ \\
\underline{Bsp:} \\ 
rechne $2590 + 4 + 4 $ in 3 stelliger dez G.P.A.
\begin{description}
\item [links]
\begin{enumerate}
\item $2590 + 4 \rightarrow 2590$
\item $2590 + 4 \rightarrow 2590$
\end{enumerate}
\item [rechts]
\begin{enumerate}
\item $4 + 4 \rightarrow 10$
\item $2590 + 10 \rightarrow 2600$
\end{enumerate}
\end{description}

\begin{center}
$\Rightarrow$ Rechenwege unterscheiden sich! \\
\end{center}\\ \\
\textit{\textbf{Regel:} beim Addieren Summanden in der Reihenfolge aufsteigender Beträge addieren.} \\ \\
Maß für der Rechenzeit eines Computers: ``flops'' floating point operations per second \\
(typisch Multiplikation oder Division) \\
(top500.org) #1 Tiake-2 3 Mio Cores, 54.000 T Flops, 17 MW\\ \\
relative Fehler wichtiger aks absoluter Fehler:\\
Näherung $\tilde{x}$ zu exaktem wert $x$, rel. fehler 
$E = \left|\frac{\tiilde{x}-x}{x}\approx\frac{\tiilde{x}-x}{\tilde{x}}$ 
für duale rechniungen am Computer B=2 $\rightarrow E_{max} = 2^{-n}$ \\
$E_{max}$ wird auch maschienenzahlgenauigkeit genannt, und gibt die kleinste potentielle Zahle an, für die gilt
$\left|E_{max}\right|$ ; $E_{max}$ kann aus Rechenergebnissen errechnet werden (ÜB1)
\newpage
\textbf{Bsp:} mit 4 mantissenziffern und Exponentenziffern
\begin{description}
\item[Addieren/Subtrahieren]
  von zahlen mit stark unterschiedlichem Exponenten: kleine Zahl kann durch Rundungsfehler verloren gehen.
  \begin{align}
    1234 + 0,5 &= 0,1234 \cdot 10^4 + 0,5 \cdot 10^0 \\
    &= 1234,5 \rightarrow 1235
    Fehler
  \end{align}
\item[Multiplikation/Division] (underflor/ oder flor möglich!)
  \begin{align}
    0,2 \cdot 10^{-5} \times 0,3 \cdot 10^{-6} &= 0,6 \cdot 10^{-12} \rightarrow 0 \\
    0,6 \cdot 10^5 \div 0,3 \cdot 10^{-6} &= 0,2 \cdot 10^{12} \rightarrow \infty
  \end{align}
\item[Fehler des Assoziativgesetzes]
  \begin{enumerate}
    \item[a)]
      \begin{align}
        x + ( y + z) &= (x + y) + z \\
        0,1111 \cdot 10^{-3} + (-0,1234 + 0,1243) &= 0,1111 \cdot 10^{-3} + 0,0009 \\
        &=0,10111 \cdot 10^{-2} \rightarrow 0,1011 \cdot 10^{-2}
      \end{align}
    \item[b)]
      \begin{align}
        (0,1111 \cdot 10^{-3} - 0,1234) + 0,1243 = 0,1233 + 0,1243 \\
        &= 0,0010 = 0,100 \cdot 10^{-2}
      \end{align}
  \end{description}
  \begin{enumerate}
    \item[a)] Fehler: $0,00001 \cdot 10^{-2} \rightarrow $ relativer fehler $\epsilon = 0,0001 = 0,01\% $
    \item[b)] Fehler: $0,00111 \cdot 10^{-2} \rightarrow $ relativer Fehler $\epsilon = 0,01 = 1\% $
  \end{enumerate}
  $\epsilon_{max} = \frac{1}{2} B^{1-4} = 0,0005$ ; im Fall b) ist $\epsilon$ also deutlich größer als $\epsilon_{max}$ !
  \subsubsection{Fehlerfortpflanzung bei Rechenoperationen}
  Fehler werden beim rechnen weitergetragen, selten werden Fehler dabei kleiner (meistens größer!). Durch Umstellen von Formeln können Fehler minimiert werden, trotzdem müssen Fehler abgeschätzt wreden.
  \begin{description}
    \item[Additionsfehler] gegeben fehlerhaste Größen $\tilde{x}$ und $\tilde{y}$ und exakten Werte $x, y$
      Fehler der Summe: $\tilde{x} + \tilde{y} - (x+y = (\tilde{x} - x) + (\tilde{y} - y)$
      Im ungünstigsten Fall addieren sich die Fehler:

      $\rightarrow$ \textit{bei Addition und Subtraktion addieren sich die Absolutbeträge der Fehler!}
    \item[Multiplikation]
      $\tilde{x} \tilde{y} - x y = \tilde{x}( \tilde{y} -y ) +\tilde{y} ( \tilde{x} -x)(\tilde{y} - y)$ \\
      also hat das Prodult von $\tilde{y}$ mit einer maschienenzahl ohne Fehler $(\tilde{x} - x$ den $\tilde{x}$-fachen Fehler (und umgekehrt); Prodult der Fehler - typischer Weise vernachlässigbar.

      $\rightarrow$ \textit{der absolute Fehler eines Prodults ist gegeben durch das Prodult des Faktors mit dem Fehler des anderen Faktors. (=2 Treme, oft ist einer der Terme dominant.)}

  Reative Fehler eines Produktes:
  \begin{equation}
    \frac{\tilde{x} \tilde{y} - x y}{\tilde{x} \tilde{y}} = \frac{\tilde{y} -y}{\tilde{y}} + \frac{\tilde{x} - x}{\tilde{x}} - \frac{(\tilde{x}-x)(\tilde{y} - y)}{\tilde{x} \tilde{y}}
  \end{equation}
  $\rightarrow$ \textit{Beim Multiplizieren addieren sich die relativen Fehler.Division analog...}
  \end{description}


  \subsubsection{Fehlerfortpflanzung -> Funktionen}
  Funktionen auswertung $f(x)$ an Stelle $\tilde{x}$ anstatt $x\, \rightarrow $ großen/kleinen Fehler von $f$.
  bei zweiten Funktionsauswertungen wird der Fehler typischerweise größer...\\
  Mittelwertsatz: $\int_x^{\tilde{x}}g\left(x'\right) dx' = g(x_0)(\tilde{x} - x)$ \\
  Mittelwert der Funktion: $ \frac{\int_x^{\tilde{x}}g\left(x'\right) dx'}{\tilde{x} - x} = $ Funktionswert $g(x_0)$
  an einer unbekannten Stelle $x_0$ im Intervall $(x, \tilde{x})$, (für stetige Funktionen $g(x)$....)\\

  wähle $g(x) = f'(x) \rightarrow \left|f(\tilde{x}) -f(x)\right| = \left|\tilde{x} -x \right| \left|f'(x_0) \right|$ \\
  $\rightarrow$ \textit{absoluter Fehler vergrößert sich für $\left|f(x_0)\right| > 1$ bzw verkleinert sich für $\left|f(x_0)\right| < 1$}

  also: Ableitung bestimmt den Verstärkungsfaktor des Fehlers!

  \underline{Abschätzung} des absoluten Fehlers: $\left|f(x) - f(\tilde{x})\right| \leq M \left|x - \tilde{x}\right|$
  mit $M = \left| f'(x_0) \right|$ \\
  Schätzung der Fehler: $\left|f(x) - f(\tilde{x}) \right| \approx \left|f'(\tilde{x})\right|\left|x - \tilde{x}\right|$

  \underline{Bsp.:}Fortpflanzung des absoluten Fehlers für $f(x) = \sin{x} \Rifgtarrow f'(x) = \cos{x}$ und damit
  $M = max_{x_0} f'(x_0) = 1$ d.h. für die meisten Argumente veringert sich der absolute Fehler! 

  \underline{Bsp.:} $f(x) = \sqrt{x} ; f'(x) = \frac{0,5}{\sqrt{x}}$divergiert also für $x \rightarrow \infty$

  \underline{relativer Fehler bei Funktionsauswertung: }%%  $\frac{f(x) - f(\tilde{x})}{\left|f(x)\right| \leg
  %% \frac{M\left x \right|}{\left|f(x)\right|} \cdot \frac{\left|x - \tilde{x} \right|}{\left| x \right| \approx
  %% \frac{\left|f'(\tilde{x})\right|\left|\tilde{x}\right|}{\left|f(\tilde{x})\right| \cdot
  %%   \frac{\left| x - \tilde{x}\right|}{\left|\tilde{x}\right|}}

  Konditionszahl: $\frac{\left|f'(\tilde{x})\right|\left|\tilde{x}\right|}{\left|f(\tilde{x})\right|}$

  Verhältnisfaktor für relative fehler; ,,qualitativ:´´ Probleme zur Koordinatenzahl $>> 1$
  ''schlech

\section{Nullstellenprobleme}
geg: stetige Funktion $f: \mathbb{R} \rightarrow \mathbb{R}$ \\
ges: Nullstelle($n$), also $x_0 e \mathbb{R}$ mit $f(x_0) = 0$ \\
grundsätzlich: 
\begin{itemize}
  \item gibt es überhaupt keine Nullstelle ?
  \item gibt es mehrere?
\end{itemize}
Zweischensatz: $f:[a,b] \rightarrow \mathbb{R}, stetig$, für $c e \mathbb{R}$ mit $f(a) \leg c \leq f(b))$ gibt
es ein $x_0 e [a,b]$ so dass $ f(x_0) = c$

für $c = 0$ ist der Satz hilfreich bei der Nullstellensuche:

suche Funktionsargumente mit unterschiedlichem Vorzeichen $f(a)f(b) < 0$ dann gibt es zwischen $a$ und $b$ mindestens eine Nullstelle!
\subsection{Bisektionsverfahren}
$f(a)f(b) < 0 = $ Nullstelle in $(a,b)$, berechne Vorzeichen von $f\left(\frac{a+b}{2}\right)$ \\
$\rightarrow f(x) = 0 in \left(0, \frac{a+b}{2}\right) $ oder $ \left(\frac{a+b}{2}, b\right)$ \\
weiter halbieren...

\underline{Bsp.:} $f(x) = x^3 - x + 0,3 = 0 $
\begin{enumerate}
  \item[a)] wie viele Nullstellen?
    $x^3-x$ hat 3 Nullstellen bei $x = \pm 1, 0$ \\
    Wir setzten also die Umgebeung von $x = \pm 1, 0 $ \\
    \begin{tabular}{r|c|c|c|l}
      x & -2 & -1 & 0,5 & 1 \\\hline
      f & -5,7 & 0,3 & -0,075 & 0,3
    \end{tabular}
\end{enumerate}
\end{document}
