\input{../../abv/lib/my}

\newcommand{\myTitle}{\LaTeX - Grundlagen}
\newcommand{\myAuthor}{Dominik Wille}
\newcommand{\myDate}{22 Oktober 2013}
\newcommand{\myTitleImage}{images/typesetting} %This Image is GNU GPL licenced.
\newcommand{\myTitleLeft}{%
   Freie Universität Berlin\\
   Zentraleinrichtung für Datenverarbeitung\\
   Betriebsysteme und Programmieren%
}
\newcommand{\myTitleRight}{%
  Dozent: \\
  Dr.\,Herbert Voß
}

\begin{document}
\myTitlepage
\section{...}
\subsection{...}
\subsection{...}
\subsection{Rundungsfehler}
Rechenoperationen mit reelen zahlen im Computer $\rightarrow$ Rundungsfehler.
\subsubsection{Gleitkommaaritmetik}
Im Vergleich zum Festpunktformat: geringerer Speicherplatzbedarf. \\ \\
$n$-stellige Gleitkommazahl, Basis $B$:
\begin{equation}
  x = \pm \left(0,z,z_2,...,z_n\right)_B \cdot B^E = \pm \sum_{i=1}^n Z_i \cdot Z_i != 0
\end{equation}
(Normalisierte Gleitkommadarstellung)
Exponent: $E e \mathbb{Z} : m <= E <= M$\\
Bsp: $+1234,567 = +(0,1234567)_{10}\cdot 10^4$ \\
($B = 10, n = 7$)
Die Werte $n,B,m,M$ maschienenabhängig (Hardware und Compiler)
Übliche Basen:
\begin{itemize}
\item $B = 2$ (Dualzahlen, im Computer)
\item $B = 8$ (Oktalzahlen)
\item $B = 10$ (Dezimal)
\item $B = 16$ (Hexdezimal)
\end{itemize}
Bsp: binäre Darstellung: 
\begin{align}
(5,0625)_{10} &= 0,50625 \cdot 10^1 \\
&= 1 \cdot 2^2 + 1 \cdot 2^0 + 0 \cdot 2^{-1} + 0 \cdot 2^{-2} + 0 \cdot 2^{-3} + 1 \cdot 2^{-4} \\
  = (101,0001)_2 &= (0,1010001)_2 \cdot 2^{3}
\end{align}
manche Zahlen lassen sich nur schwer als Dualzahlen darstellen:\\
\begin{itemize} 
\item $(3)_{10} = (11)_2$ geht
\item $(0,3)_{10} = 0 \cdot 2^{-1} + 1 \cdot 2^{-2} + ... = (0,010011001...)_2$ geht nicht
\end{itemize}
\textbf{Genauigkeit der Darstellung} \\ \\
23 Stellen $\Roghtarrow$ $11111111111111111111111 = 2^{23} - 1 = 8.388.608$ \\
$\Rightarrow$ 6 Ziffren können unterschieden werden. \\ \\
52 Stellen $2^{52} = 4.503.599.627.370.496$\\
$\Rightarrow$ 15 Stellen können unterschienden werden.
Die größte darstellbare Zahl entspricht der größten Maschienenzahl.
\begin{align}
x_{max} = (0,[B-1][B-1]...[B-1])_B \cdot B^M = (1-B^{-n}) \cdot B^M 
\end{align}
kleinste darstellbare Zahl
\begin{align}
x_{min} = (0,1000000)_B \cdot B^m = (1-B^{-n}) \cdot B^{m-1} 
\end{align}
\begin{center}
  $\Rightarrow$ \textit{Die menge der Maschienenzahlen ist endlich}
\end{center}
\underline{Bsp:}\\
$x_{max} + x_{max} = \infty$ \\
$x_{min} \cdot B^{-1} = 0
\subsubsection{Rundungsfehler}
Beim runden einer Zahl $x$ wird eine Näherung $rd(x)$ unter den Maschienenzahlen geliefert,  so dass der absolute Fehler $\left|x-rd(x)\right|$ minimal ist, der unvermeidbare Fahler ist der Rundungsfehler. Eine $n$-stellige Dezimalzahl im Gleitkommaformat
\begin{align}
  x = \pm (0,z_1,...,z_n)_{10}=rd(x)
\end{align}
hat einen maximalen absoluten Fehler : 
\begin{align}
  \left| x - rd(x)\right| &<= 0,000..005 \cdot 10^E \\
  &= 0,5 \cdot E^{E-n}
\end{align}
, für allgemeine Basis $B$:
\begin{align}
  \left|x-rd(x)\right| <= \frac{B \cdot 1}{2 \cdot B} B^{E-n} = \frac{1}{2} B^{E-n}
\end{align}
\underline{Rundungsfehler werden durch die rechnung getragen!} \\ \\
$n$-stellige Gleitkommaaritmetik: \\
jede einzelne Rechenoperation ($+,-,\times,\div$)wird auf $n+1$ Stellen genau berechnet und dann auf n stellen gerundet. Jedes Zwischenergebnis, nicht Endergebnis! \\ \\
\underline{Bsp:} \\ 
rechne $2590 + 4 + 4 $ in 3 stelliger dez G.P.A.
\begin{description}
\item [links]
\begin{enumerate}
\item $2590 + 4 \rightarrow 2590$
\item $2590 + 4 \rightarrow 2590$
\end{enumerate}
\item [rechts]
\begin{enumerate}
\item $4 + 4 \rightarrow 10$
\item $2590 + 10 \rightarrow 2600$
\end{enumerate}
\end{description}

\begin{center}
$\Rightarrow$ Rechenwege unterscheiden sich! \\
\end{center}\\ \\
\textit{\textbf{Regel:} beim Addieren Summanden in der Reihenfolge aufsteigender Beträge addieren.} \\ \\
Maß für der Rechenzeit eines Computers: ``flops'' floating point operations per second \\
(typisch Multiplikation oder Division) \\
(top500.org) #1 Tiake-2 3 Mio Cores, 54.000 T Flops, 17 MW\\ \\
relative Fehler wichtiger aks absoluter Fehler:\\
Näherung $\tilde{x}$ zu exaktem wert $x$, rel. fehler 
$E = \left|\frac{\tiilde{x}-x}{x}\approx\frac{\tiilde{x}-x}{\tilde{x}}$ 
für duale rechniungen am Computer B=2 $\rightarrow E_{max} = 2^{-n}$ \\
$E_{max}$ wird auch maschienenzahlgenauigkeit genannt, und gibt die kleinste potentielle Zahle an, für die gilt
$\left|E_{max}\right|$ ; $E_{max}$ kann aus Rechenergebnissen errechnet werden (ÜB1)
\end{document}
